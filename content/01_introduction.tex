\chapter{Introduction}
The topic of this thesis belongs to an astrophysical subsection investigating some of the most extreme phenomena in the universe.
For example supernovae and black holes are known to be sources of high energetic particles,
but the events emitting this radiation are still a subject of research.
To deepen the knowledge concerning these events the emitted gamma rays and hadrons can be examined by telescopes.

The advantage of gamma rays over likewise emitted hadrons is their indivertible nature against electromagnetic fields.
Whereas the direction of hadrons is scrambled by magnetic fields while passing galaxies and nebulae, gamma rays are unresponsive.
When the sub-atomic particles arrive at the earth the direction of the gamma rays is still preserved
and can be used to detect and measure their source.

When entering the aerosphere gamma rays as well as hadrons collide with other particles
and transfer some of their energy to their collision partners.
Both particles cause air showers which emit Cherenkov radiation by reason of the high energies involved.
Ground-level telesopes can measure the gamma rays and hadrons indirectly by observing the brief Cherenkov flash.

Afterwards machine learning algorithms classify the recorded images as being triggered by a gamma ray or a hadron.
Using only the measurements caused by gamma rays their preserved direction displays their cosmic source.
Further investigation of these gamma events can reveal source specific characteristics.

This thesis will take a look into Deep Learning and Convolutional Neural Networks
to draw level or outperform the currently used classification algorithm.
Therefor different network architectures will be implemented in Google's python library tensorflow.
These networks are being evaluated and their hyperparameters will be optimized.
Finally the best CNN model will be analysed and compared to the currently used classifier.
