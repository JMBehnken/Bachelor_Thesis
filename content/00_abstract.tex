\thispagestyle{plain}

\section*{Abstract}
High energetic gamma rays from cosmic sources reach the aerosphere and interact with it.
Thereby caused particle showers emit Cherenkov radiation, which can be recorded by telescopes at ground-level.

This thesis is motivated by the study of these pictures from the FACT telescope with Deep Learning.
To identify the source of the gamma rays a gamma hadron separation will be realised.
Therefor a simulated dataset is used to train a Convolutional Neural Network (CNN).
To come close to the ideal performance 15 networkarchitectures and regularizations are beeing compared to each other.

Afterwards a comparison is drawn between the currently used classifier and the CNN.
Although the CNN performs comparable on a testdataset, it clearly fails with real measured images.
This can be reduced to a systematic error between simulated and real datasets (Monte Carlo Mismatch)

\section*{Kurzfassung}
\begin{german}
Von kosmischen Quellen erreicht energiereiche Gammastrahlung die obersten Schichten der Erdatmosphäre und wechselwirkt mit dieser.
Dabei entstehende Teilchenschauer senden Tscherenkov-Strahlung aus, welche von Teleskopen am Erdboden aufgenommen werden kann.

Motivation für diese Arbeit ist die Untersuchung dieser Bilder vom FACT-Teleskop mittels Deep Learning.
Um die Quellen der Gammastrahlung ausfindig zu machen, wird eine Gamma-Hadron-Separation durchgeführt.
Hierfür wird auf simulierten Datensätzen ein Convolutional Neural Network (CNN) trainiert.
Um der optimalen Performance nahe zu kommen werden 15 Netzwerkarchitekturen und Regularisierungen miteinander verglichen.

Anschließend wird ein Vergleich zwischen dem bisher eingesetzten Klassifizierer und dem CNN gezogen.
Obwohl das CNN auf Testddaten eine vergleichbar gute Performance erreicht, zeigt sich ein deutliches Versagen auf real gemessenen Bildern.
Dies wird auf den systematischen Fehler zwischen simulierten und echten Daten zurückgeführt (Monte Carlo Mismatch).
\end{german}
