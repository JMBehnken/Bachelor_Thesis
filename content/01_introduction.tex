\chapter{Introduction}
The topic of this thesis falls under an area of astrophysics, investigating some of the most extreme phenomena in the universe.
For example, supernovae and black holes are known to be sources of high energy particles,
but the processes emitting this radiation are still a subject of research.
To deepen the knowledge concerning these processes the emitted gamma rays and hadrons can be examined by telescopes.

Whereas the direction of hadrons is scrambled by magnetic fields while passing through galaxies and nebulae,
gamma rays are unaffected by such fields and propagate in straight lines.
When the sub-atomic particles arrive at the Earth, the direction of the gamma rays is preserved
and can be used to detect their source and measure its properties.

When entering the atmosphere both gamma rays and hadrons collide with other particles,
transferring some of their energy to their collision partners.
Both types of particles cause air showers which emit Cherenkov radiation by reason of the high energies involved.
Ground-level telescopes can measure the gamma rays and hadrons indirectly by observing the brief Cherenkov flash.

Afterwards, machine learning algorithms classify the recorded images as being induced by a gamma ray or a hadron.
Using only the measurements caused by gamma rays, their preserved direction reveals their cosmic source.
Further investigation of these gamma events can reveal source specific characteristics.

This thesis will apply Deep Learning and Convolutional Neural Networks to FACT images and
will compare the network's performance to the currently-used classification algorithm's performance.
To achieve this, different network architectures will be implemented in Google's Python library \texttt{TensorFlow}.
Afterwards the network's hyperparameters are optimized and these networks are evaluated.
Finally, the best CNN model will be used to analyse the Crab nebula observations
and the results will be compared to the currently-used Random Forest.
